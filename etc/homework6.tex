\documentclass[10pt]{article}

\usepackage[problem]{handout}
\usepackage{alltt}
\usepackage{hyperref}

\setlength{\unitlength}{2pt}

% latex2html wrecks \framebox commands in pictures
% so we give it a new name in a way that latex2html 
% doesn't understand so it will leave them alone.
\let\FrameBox=\framebox

\class{CompSci 351 / CompSt 751: Data Structures \& Algorithms}
\period{Spring 2023}
\title[Homework \#6]{Homework \# 6 \\ 
\textbf{due Monday, March 6, 7:00 PM }}


\begin{document}
\maketitle

This assignment will focus on the concepts of generics,
and doubly-linked lists with dummy nodes in a collection class.

Use this link to accept the assignment:
\begin{quote}
\url{INVITE_LINK}
\end{quote}

\section{The Data Structure}

For this homework, you will implement a \emph{generic} implementation of
the standard collection interface using a \emph{cyclic doubly-linked
list with a dummy node}.  The fields of the class should be exactly
the following:
\begin{description}
\item[dummy] The dummy node (never null).
\item[count] The number of elements in the collection.
\item[version] A version stamp to implement fail-fast semantics.
\end{description}
The nodes should be doubly linked: every link in the forward direction
(\texttt{p.link == q}) should be reflected by a link in the reverse
direction (\texttt{q.prev == p}).  There should be no null pointers in
these links, which makes the programming simpler.
(The data in the nodes can be null, of course.)
The data field of the dummy node should be assigned the ``impossible''
value of itself, that is (\texttt{dummy.data == dummy}) should be
true.  None of the non-dummy nodes should have this data value.
Finally the ``count'' field should be one fewer than the number of
nodes in the (cyclic) list.

The list contains a \emph{dummy} node even when it represents an empty
collection.  
At the start, when the collection is empty, the \verb|next| and \verb|prev|
fields of the dummy node will point back to the dummy node.  After elements are
added, the dummy node's \verb|next| pointer refers to the first (non-dummy) node in the list
and the \verb|prev| pointer refers to the last node in the list.

\section{Generics}

In some earlier programs, you may have \emph{used} generic classes.
This week you will be \emph{creating} your own generic class,
\textsf{LinkedCollection}.
The purpose of this is to familiarize you with making classes that
are not limited to a specific element type.  
Inside the generic class, you can use type parameter \verb|E|
everywhere where an explicit type would be used (e.g., \verb|Transaction|
in Homework \#~4).
Furthermore, since the \verb|Node| class is also generic, you must
create instances of \textsf{Node} using this parameter as in
\verb|new Node<E>(...)|.

\section{The methods}

As usual, you should only override the methods that need to be
overridden for Java reasons (``required''), correctness
(``implementation'') or efficiency.

In this homework, you will find
that one of the methods implemented by \texttt{AbstractCollection}
\emph{almost always} does the right thing, but 
needs to be overridden for a special case.  Until this is done, you will fail one of the
tests with a stale iterator problem.  In your overriding, except for
the special case, you should
defer to the \texttt{AbsractCollection} implementation using the
syntax  \texttt{super.methodName(\ldots)} to do the work.  This kind
of overriding should be commented as \texttt{// decorate} because it
only adds a little bit to the superclass implementation.

Don't use any auxiliary structures (arrays or other collections) to do
the work of any methods.  We also don't require, or even recommend that
you implement a \texttt{clone()} method.

You \emph{may} optionally override the \texttt{toString} method for
debugging purposes.  We have no tests of \texttt{toString} this time.

\section{The Iterator}

As with Homework \#3, the collection's iterator will be ``addable.''
Since it is a nested \emph{non}-static class, it should \emph{not}
have its own generic parameter.  Instead it will use the outer class'
generic type parameter.  It will implement \texttt{Iterator} with the
generic parameter from the outer class.

The iterator should have the following data structure:
\begin{description}
  \item[cursor] The pointer to the node with the element that was last
    returned by ``next'' until/unless that element is removed, in
    which case it moves back to the previous node.  In all cases, the
    ``next'' element (if any) will be the one in the node \emph{after} the
    cursor.  When the iterator is initialized, the ``cursor'' starts
    on the dummy node.  If the node after the cursor is the dummy node,
    then there is no next element.
  \item[isCurrent] This boolean is true if the cursor is pointing to a
    node with the current element (the one that can be removed).  It
    should never be true that the dummy's data could be removed.
  \item[colVersion] The version of the collection that this iterator
    can work with.  It is used to implement fail-fast semantics.
\end{description}
You will need to implement \texttt{wellFormed} for the iterator (as
well as for the main class) that checks this data structure.  You
should be able to figure out the conditions.  Make sure to follow
the convention we gave you in Homework \#3---checking the outer
invariant first and then checking for version mismatch.  Remember that
if the version does not match, the the invariant is assumed to be true.

\section{What You Need To Do}

We recommend you proceed in the following manner:
\begin{enumerate}
  \item Add the required data structure fields, methods and nested
    classes to
    \texttt{LinkCollection.java} so that all errors are gone.
    In particular, the \texttt{Spy} class should compile without
    errors, and without being changed.
  \item Write the \texttt{wellFormed} methods, using
    \texttt{TestInvariant}.
  \item Implement methods needed to pass the \texttt{TestDirect} and
    \texttt{TestLinkedCollection} tests.  (Recall that
    \texttt{TestCollection} is an abstract class and should not be
    run.)
  \item Make sure that every method you add is either
    \verb|@Override| with a reason (one of the four: required,
    implementation, efficiency or decorate) or has a full
    documentation comment (with \verb|@param|,
    \verb|@exception| and \verb|@return| parts if appropriate). 
  \item Use random testing to find any remaining implementation
    errors.
  \item Use efficiency testing to determine any additional overridings
    that may be necessary.  After passing efficiency testing, check
    that no bugs have been introduced by running random testing again.
\end{enumerate}
There are no locked tests in this assignment.

Your class should use the same style as in the previous homework
assignments: it should assert the correctness of the invariant
at the beginning and end of any method that could change the state
of the data structure, and at the beginning of every public method.
The invariant should also be checked at the end of the
constructor and when the iterator invariant is checked.

Remember to update version numbers whenever the collection changes the
number of elements.

\section{Files}

The homework repository
contains the following files:
\begin{description}
\item[src/TestLinkedCollection.java] This driver tests the collection operations.
\item[src/TestDirect.java] 
\item[src/TestInvariant.java] Test driver for invariants of list and iterator.
\item[src/TestEfficiency.java] Test driver that runs with larger data sets.
\item[src/edu/uwm/cs351/LinkedCollection.java] This is the skeleton file you should work with.
\end{description}

\end{document}

% LocalWords:  CS TreatList cc cs cp cs src SKEL turnin myworld ostream const
% LocalWords:  href html LocalWords ADT bool int ItemType cerr endl rethrow
% LocalWords:  SortedList prev LinkedCollection AbstractCollection





% LocalWords:  methodName superclass toString isCurrent boolean
% LocalWords:  colVersion wellFormed
